\documentclass{article}

\usepackage{fancyhdr}
\usepackage{extramarks}
\usepackage{amsmath}
\usepackage{amsthm}
\usepackage{amssymb}
\usepackage{amsfonts}
\usepackage{tikz}
\usepackage[plain]{algorithm}
\usepackage{algpseudocode}

%
% Basic Document Settings
%

\topmargin=-0.45in
\evensidemargin=0in
\oddsidemargin=0in
\textwidth=6.5in
\textheight=9.0in
\headsep=0.25in

\linespread{1.2}

\pagestyle{fancy}
\rhead{\hmwkAuthorName}
\lhead{\hmwkClass: \hmwkTitle}


\renewcommand\headrulewidth{0.4pt}
\renewcommand\footrulewidth{0.4pt}
\renewcommand{\proofname}{\textit{\textbf{ Proof.}}}

% \setlength\parindent{0pt}

%
% Create Problem Sections
%

\newcommand{\enterProblemHeader}[1]{
    \nobreak\extramarks{}{Problem \arabic{#1} continued on next page\ldots}\nobreak{}
    \nobreak\extramarks{Problem \arabic{#1} (continued)}{Problem \arabic{#1} continued on next page\ldots}\nobreak{}
}

\newcommand{\exitProblemHeader}[1]{
    \nobreak\extramarks{Problem \arabic{#1} (continued)}{Problem \arabic{#1} continued on next page\ldots}\nobreak{}
    \stepcounter{#1}
    \nobreak\extramarks{Problem \arabic{#1}}{}\nobreak{}
}

\setcounter{secnumdepth}{0}
\newcounter{partCounter}
\newcounter{homeworkProblemCounter}
\setcounter{homeworkProblemCounter}{1}
\nobreak\extramarks{Problem \arabic{homeworkProblemCounter}}{}\nobreak{}

%
% Homework Problem Environment
%
% This environment takes an optional argument. When given, it will adjust the
% problem counter. This is useful for when the problems given for your
% assignment aren't sequential. See the last 3 problems of this template for an
% example.
%
\newenvironment{homeworkProblem}[1][-1]{
    \ifnum#1>0
        \setcounter{homeworkProblemCounter}{#1}
    \fi
    \section{Problem \arabic{homeworkProblemCounter}}
    \setcounter{partCounter}{1}
    \enterProblemHeader{homeworkProblemCounter}
}{
    \exitProblemHeader{homeworkProblemCounter}
}


\newcommand{\hmwkTitle}{Assignment 10}
\newcommand{\hmwkClass}{Manifold Learning and Sparse Representation}
\newcommand{\hmwkAuthorName}{\textbf{ZHANG Yuan}, 1601111332 }
\date{}
%
% Title Page
%

\title{
    \textmd{\textbf{\hmwkClass:\ \hmwkTitle}}\\
}

\author{\hmwkAuthorName}

\renewcommand{\part}[1]{\textbf{\large Part \Alph{partCounter}}\stepcounter{partCounter}\\}

%
% Various Helper Commands
%

% Useful for algorithms
\newcommand{\alg}[1]{\textsc{\bfseries \footnotesize #1}}

% For derivatives
\newcommand{\deriv}[1]{\frac{\mathrm{d}}{\mathrm{d}x} (#1)}

% For partial derivatives
\newcommand{\pderiv}[2]{\frac{\partial}{\partial #1} (#2)}

% Integral dx
\newcommand{\dx}{\mathrm{d}x}

% Alias for the Solution section header
\newcommand{\solution}{\textbf{\large Solution}}

% Probability commands: Expectation, Variance, Covariance, Bias
\newcommand{\E}{\mathrm{E}}
\newcommand{\Var}{\mathrm{Var}}
\newcommand{\Cov}{\mathrm{Cov}}
\newcommand{\Bias}{\mathrm{Bias}}

\begin{document}

\maketitle
\section{Exercise 165}
The mutual-coherence is 0.9988 (the normalized inner product of the last two columns).
\section{Exercise 167}
We first compute $\tilde A$ with normalized columns. Then compute Gram matrix $G = \tilde A^T\tilde A$ of $\tilde A$ and then sort every row of $G$ in descending order to obtain $G_S$. Thus,
$$
\mu_1(2) = max_{1\leq j \leq 4}G_S(j,2)+G_S(j,3) = 0.9988    0.9981 = 1.9969.
$$
\section{Exercise 168}
\begin{proof}
 We can take any $p$ columns of A to construct a sub-matrix $A_p$ and obtain the Gram matrix $G_p$ of $A_p$. Since $\mu_1(p-1) < 1$ deduces to the sum of off-diagonal of elements of each column is strict less than 1 (let one column equal to $j$ and the other $\Lambda$), $G_p$ is strictly diagonally dominant with positive diagonal elements (which is 1) and hence positive definite. Therefore, given $\mu_1(p-1) < 1$, any $p$ columns of A are linear independent and $spark(A)>p$. In other words, $spark(A)$ should be large or equal than the smallest p such that $\mu_1(p-1) \leq 1$.
\end{proof}

\section{Exercise 169}
The uncertainty and uniqueness properties follow immediately with the lower bound of spark given in the last exercise.

\textbf{Uncertainty.} $\Vert x_1 \Vert_0 + \Vert x_1 \Vert_1 \geq  \min{p\vert \mu_1(p-1)\geq1}$.

\textbf{Uniqueness.} If a system of linear equations Ax = b has a solution x obeying $\Vert x \Vert_0 < \frac{1}{2}\min\{p\vert \mu_1(p-1)\geq1\}$, this solution is necessarily the sparsest possible.

\end{document}