\documentclass{article}

\usepackage{fancyhdr}
\usepackage{extramarks}
\usepackage{amsmath}
\usepackage{amsthm}
\usepackage{amssymb}
\usepackage{amsfonts}
\usepackage{tikz}
\usepackage[plain]{algorithm}
\usepackage{algpseudocode}

%
% Basic Document Settings
%

\topmargin=-0.45in
\evensidemargin=0in
\oddsidemargin=0in
\textwidth=6.5in
\textheight=9.0in
\headsep=0.25in

\linespread{1.2}

\pagestyle{fancy}
\rhead{\hmwkAuthorName}
\lhead{\hmwkClass: \hmwkTitle}


\renewcommand\headrulewidth{0.4pt}
\renewcommand\footrulewidth{0.4pt}
\renewcommand{\proofname}{\textit{\textbf{ Proof.}}}

% \setlength\parindent{0pt}

%
% Create Problem Sections
%

\newcommand{\enterProblemHeader}[1]{
    \nobreak\extramarks{}{Problem \arabic{#1} continued on next page\ldots}\nobreak{}
    \nobreak\extramarks{Problem \arabic{#1} (continued)}{Problem \arabic{#1} continued on next page\ldots}\nobreak{}
}

\newcommand{\exitProblemHeader}[1]{
    \nobreak\extramarks{Problem \arabic{#1} (continued)}{Problem \arabic{#1} continued on next page\ldots}\nobreak{}
    \stepcounter{#1}
    \nobreak\extramarks{Problem \arabic{#1}}{}\nobreak{}
}

\setcounter{secnumdepth}{0}
\newcounter{partCounter}
\newcounter{homeworkProblemCounter}
\setcounter{homeworkProblemCounter}{1}
\nobreak\extramarks{Problem \arabic{homeworkProblemCounter}}{}\nobreak{}

%
% Homework Problem Environment
%
% This environment takes an optional argument. When given, it will adjust the
% problem counter. This is useful for when the problems given for your
% assignment aren't sequential. See the last 3 problems of this template for an
% example.
%
\newenvironment{homeworkProblem}[1][-1]{
    \ifnum#1>0
        \setcounter{homeworkProblemCounter}{#1}
    \fi
    \section{Problem \arabic{homeworkProblemCounter}}
    \setcounter{partCounter}{1}
    \enterProblemHeader{homeworkProblemCounter}
}{
    \exitProblemHeader{homeworkProblemCounter}
}


\newcommand{\hmwkTitle}{Assignment 13}
\newcommand{\hmwkClass}{Manifold Learning and Sparse Representation}
\newcommand{\hmwkAuthorName}{\textbf{ZHANG Yuan}, 1601111332 }
\date{}
%
% Title Page
%

\title{
    \textmd{\textbf{\hmwkClass:\ \hmwkTitle}}\\
}

\author{\hmwkAuthorName}

\renewcommand{\part}[1]{\textbf{\large Part \Alph{partCounter}}\stepcounter{partCounter}\\}

%
% Various Helper Commands
%

% Useful for algorithms
\newcommand{\alg}[1]{\textsc{\bfseries \footnotesize #1}}

% For derivatives
\newcommand{\deriv}[1]{\frac{\mathrm{d}}{\mathrm{d}x} (#1)}

% For partial derivatives
\newcommand{\pderiv}[2]{\frac{\partial}{\partial #1} (#2)}

% Integral dx
\newcommand{\dx}{\mathrm{d}x}

% Alias for the Solution section header
\newcommand{\solution}{\textbf{\large Solution}}

% Probability commands: Expectation, Variance, Covariance, Bias
\newcommand{\E}{\mathrm{E}}
\newcommand{\Var}{\mathrm{Var}}
\newcommand{\Cov}{\mathrm{Cov}}
\newcommand{\Bias}{\mathrm{Bias}}

\begin{document}

\maketitle

\section{Exercise 250.}
\begin{proof}
An easier proof than ref. [312] is offered as follows:

Obviously, the objective function is strictly convex with regard to $Z$. Thus, there exists a unique minimizer and we only have to verify that Eq. (11.2) is a minimizer. 

We know that $\hat{Z} = U\Sigma_{\lambda}U^{T}$ is a minimizer iff. 

$$
0 \in \partial(\frac{1}{2}\Vert \phi(D) - \phi(D)\hat{Z} \Vert_F^2 + \lambda\Vert \hat{Z} \Vert_{*}) \Leftrightarrow \frac{1}{\lambda}K(I - \hat{Z}) \in   \partial \Vert  \hat{Z} \Vert_{*},
$$
where $K = \phi(D)\phi(D)^{T} = U \Sigma U^{T}$. This proposition holds due to the following fact
\begin{align*}
    \frac{1}{\lambda}K(I - \hat{Z}) &=\frac{1}{\lambda} U \Sigma U^{T} ( I - U\Sigma_{\lambda}U^{T}) \\
    &=  \frac{1}{\lambda} U \Sigma (I - \Sigma_{\lambda}) U^{T} \\
    &= \frac{1}{\lambda} U \Sigma \text{diag}(\{\frac{\lambda}{\sigma_i}\vert \sigma_i > \lambda\},I) U^{T} \\
    &= U \text{diag}(I,\{\frac{\sigma_i}{\lambda}\vert \sigma_i \leq \lambda \}) U^T \\
    &= U_0U_0^{T} + U_1 W U_1^T,
\end{align*}
where $W = \text{diag}(\{\frac{\sigma_i}{\lambda}\vert \sigma_i \leq \lambda \}) = \text{diag}(\{\frac{\sigma_i}{\lambda}\vert \frac{\sigma_i}{\lambda} \leq 1 \})$ and $U_0$ ($U1$) consists of eigenvectors corresponding to those eigen-values that are larger (less) than $\lambda$. We can verify that $\Vert W \Vert_2 \leq 1$ and hence $\frac{1}{\lambda}K(I - \hat{Z}) \in   \partial \Vert  \hat{Z} \Vert_{*}$ thanks to Theorem 126.
\end{proof}

\end{document}

