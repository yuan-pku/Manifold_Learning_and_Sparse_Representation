\documentclass{article}

\usepackage{fancyhdr}
\usepackage{extramarks}
\usepackage{amsmath}
\usepackage{amsthm}
\usepackage{amssymb}
\usepackage{amsfonts}
\usepackage{tikz}
\usepackage[plain]{algorithm}
\usepackage{algpseudocode}

%
% Basic Document Settings
%

\topmargin=-0.45in
\evensidemargin=0in
\oddsidemargin=0in
\textwidth=6.5in
\textheight=9.0in
\headsep=0.25in

\linespread{1.2}

\pagestyle{fancy}
\rhead{\hmwkAuthorName}
\lhead{\hmwkClass: \hmwkTitle}


\renewcommand\headrulewidth{0.4pt}
\renewcommand\footrulewidth{0.4pt}
\renewcommand{\proofname}{\textit{\textbf{ Proof.}}}

% \setlength\parindent{0pt}

%
% Create Problem Sections
%

\newcommand{\enterProblemHeader}[1]{
    \nobreak\extramarks{}{Problem \arabic{#1} continued on next page\ldots}\nobreak{}
    \nobreak\extramarks{Problem \arabic{#1} (continued)}{Problem \arabic{#1} continued on next page\ldots}\nobreak{}
}

\newcommand{\exitProblemHeader}[1]{
    \nobreak\extramarks{Problem \arabic{#1} (continued)}{Problem \arabic{#1} continued on next page\ldots}\nobreak{}
    \stepcounter{#1}
    \nobreak\extramarks{Problem \arabic{#1}}{}\nobreak{}
}

\setcounter{secnumdepth}{0}
\newcounter{partCounter}
\newcounter{homeworkProblemCounter}
\setcounter{homeworkProblemCounter}{1}
\nobreak\extramarks{Problem \arabic{homeworkProblemCounter}}{}\nobreak{}

%
% Homework Problem Environment
%
% This environment takes an optional argument. When given, it will adjust the
% problem counter. This is useful for when the problems given for your
% assignment aren't sequential. See the last 3 problems of this template for an
% example.
%
\newenvironment{homeworkProblem}[1][-1]{
    \ifnum#1>0
        \setcounter{homeworkProblemCounter}{#1}
    \fi
    \section{Problem \arabic{homeworkProblemCounter}}
    \setcounter{partCounter}{1}
    \enterProblemHeader{homeworkProblemCounter}
}{
    \exitProblemHeader{homeworkProblemCounter}
}


\newcommand{\hmwkTitle}{Assignment 3}
\newcommand{\hmwkClass}{Manifold Learning and Sparse Representation}
\newcommand{\hmwkAuthorName}{\textbf{ZHANG Yuan}, 1601111332 }
\date{}


%
% Title Page
%

\title{
    \textmd{\textbf{\hmwkClass:\ \hmwkTitle}}\\
}

\author{\hmwkAuthorName}

\renewcommand{\part}[1]{\textbf{\large Part \Alph{partCounter}}\stepcounter{partCounter}\\}

%
% Various Helper Commands
%

% Useful for algorithms
\newcommand{\alg}[1]{\textsc{\bfseries \footnotesize #1}}

% For derivatives
\newcommand{\deriv}[1]{\frac{\mathrm{d}}{\mathrm{d}x} (#1)}

% For partial derivatives
\newcommand{\pderiv}[2]{\frac{\partial}{\partial #1} (#2)}

% Integral dx
\newcommand{\dx}{\mathrm{d}x}

% Alias for the Solution section header
\newcommand{\solution}{\textbf{\large Solution}}

% Probability commands: Expectation, Variance, Covariance, Bias
\newcommand{\E}{\mathrm{E}}
\newcommand{\Var}{\mathrm{Var}}
\newcommand{\Cov}{\mathrm{Cov}}
\newcommand{\Bias}{\mathrm{Bias}}

\begin{document}

\maketitle

%\pagebreak
\section{Exercise 46.}
The results are listed in Table 1. (The code in Matlab is attached.) 
\begin{table}[h]
\centering
\begin{tabular}{c|c}
\hline
Euclidean distance&0.734042\\
\hline
l1 distance&0.668856\\
\hline
Angular distance&0.773333\\
\hline
Cosine distance&\textbf{0.645510}\\
\hline
\end{tabular}
\end{table}

Thus, \textit{Cosine distance} offers the best dissimilarity metrics in terms of classification.
\section{Exercise 48.}
\subsection{Prove $\frac{1}{2}\sum_{i,j} W_{ij}\Vert \mathbf{f_i-f_j}\Vert^2 = tr(\mathbf{F}\mathbf{L}_{W}\mathbf{F}^T)$, where $\mathbf{F} =(\mathbf{f_1,...,f_n})$. }
\begin{proof}
\begin{align*}
\frac{1}{2}\sum_{i,j} W_{ij} \Vert\mathbf{f_i-f_j}\Vert^2 &= 
\frac{1}{2}\sum_{i,j}  W_{ij} (\mathbf{f_i-f_j})^T (\mathbf{f_i-f_j}) \\
&= \frac{1}{2}\sum_{i,j}  W_{ij} (\mathbf{f_i^Tf_i+f_j^Tf_j-f_{i}^Tf_{j}-f_{j}^Tf_{i}}) \\
&= \sum_{i} (\sum_{j} W_{ij})\mathbf{f_{i}}^T \mathbf{f_{i}} - \sum_{i,j} W_{ij}\mathbf{f_{j}}^T \mathbf{f_{i}}\\
&= \sum_{i} d_i\mathbf{f_{i}}^T \mathbf{f_{i}} - \sum_{i,j} W_{ij}\mathbf{f_{j}}^T \mathbf{f_{i}}\\
&= tr(\sum_{i}d_{i}\mathbf{f_{i}}^T \mathbf{f_{i}} ) - tr(\sum_{i,j} W_{ij}\mathbf{f_{j}}^T \mathbf{f_{i}})\\
&=tr(\sum_{i}\mathbf{f_{i}}d_{i}\mathbf{f_{i}}^T ) - tr(\sum_{i,j} \mathbf{f_{i}}W_{ij}\mathbf{f_{j}}^T) \\
&=tr(\mathbf{FD}_W\mathbf{F}^T) - tr(\mathbf{FWF}^T)\\
&=tr(\mathbf{F(D}_W-\mathbf W)\mathbf{F}^T)\\
&=tr(\mathbf{FL}_{W}\mathbf{F}^T)
\end{align*}
\end{proof}
\subsection{Derive the Laplacian matrix of $\frac{1}{2}\sum_{i,j} W_{ij}(f_i-f_j)^2$ under the constrain $\sum_{i}W_{ii}f_i^2 = 1$.}
Let $\mathbf{g} \triangleq \mathbf D_{W}^{\frac{1}{2}} \mathbf f$ so that the constrain $\sum_{i}W_{ii}f_i^2=1$ deduces to $\Vert \mathbf g \Vert = 1$. Then, we have
\begin{align*}
\frac{1}{2}\sum_{i,j} W_{ij}(f_i-f_j)^2 & = \sum_{i}W_{ii}f_i^2 - \sum_{i,j} W_{ij}f_if_j \\
& = 1 - \mathbf{f}^T \mathbf W \mathbf{f} = \mathbf g^T  \mathbf g - \mathbf g^T \mathbf D^{-\frac{1}{2}}\mathbf W\mathbf D^{-\frac{1}{2}} \mathbf g \\
&= \mathbf g^T (\mathbf I - \mathbf D^{-\frac{1}{2}}\mathbf W\mathbf D^{-\frac{1}{2}})\mathbf g^T \\
&\triangleq \mathbf g^T \mathbf{\tilde L}_W \mathbf g^T 
\end{align*}
where the Laplaceian matrix $\mathbf{\tilde L}_W = \mathbf I - \mathbf D^{-\frac{1}{2}}\mathbf W\mathbf D^{-\frac{1}{2}}$.
\end{document}